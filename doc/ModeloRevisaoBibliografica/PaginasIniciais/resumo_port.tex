
\hspace{-7mm}{\Large{\bf Resumo }} \\ [6mm]

Esse trabalho apresenta uma discuss\~ao te\'orica e os resultados obtidos a partir da implementa\c{c}\~ao de algoritmos de inser\c{c}\~ao aplicados ao problema de triangula\c{c}\~ao de Delaunay no plano. A representa\c{c}\~ao de todas as fases do processo construtivo  \'e baseada na estrutura de dados topol\'ogica \textit{winged edge modificada}. Dentre todas as triangula\c{c}\~oes existentes e associadas a um mesmo conjunto de pontos, a triangula\c{c}\~ao de Delaunay \'e aquela que n\~ao s\'o maximiza o menor dos \^angulos, mas garante sob certas condi\c{c}\~oes, a unicidade da triangula\c{c}\~ao. Nesse sentido, a triangula\c{c}\~ao de Delaunay associada a um conjunto de pontos em $R^2$, apresenta caracter\'isticas de regularidade local e global, raz\~ao pela qual \'e indispens\'avel para um amplo conjunto de aplica\c{c}\~oes, como por exemplo, modelagem digital de terrenos e gera\c{c}\~ao autom\'atica de malhas de elementos finitos.

Os resultados previamente obtidos serviram como ponto de partida para que um projeto associado ao diagrama de Voronoi, que \'e o dual da triangulação de Delaunay, pudesse ser obtido de maneira indireta.

\vspace{15mm}

\hspace{-7mm}{\Large {\bf Palavras-Chave:}}\\

\hspace{-7mm}
Triangula\c{c}\~{a}o de Delaunay \\ [2mm]
Estruturas de dados topol\'{o}gicas \\ [2mm]
\textit{winged-edge modificada} \\ [2mm]





%%% Local Variables: 
%%% mode: latex
%%% TeX-master: "tese"
%%% End: 
