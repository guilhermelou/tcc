


\chapter{Introdu��o}

Essa monografia apresenta uma fundamenta��o...

Conv�m real�ar que o desenvolvimento do projeto me ajudou a ter um contato com ferramentas que colaboraram para a edi��o do presente documento, entre elas o \textit{Xfig}(editor de figuras vetoriais) e o \LaTeX(processador de textos cient�fico), al�m de um contato direto com o sistema operacional Linux que foi o ambiente suporte para as ferramentas e me ajudou a ter uma no��o b�sica da sua opera��o e poder� servir para aprofundar meu conhecimento para uso em outras aplica��es que pretendo desenvolver futuramente.

\section{Objetivos do Trabalho}

Esse projeto tem como objetivo principal...


\section{Organiza��o do Trabalho}

O cap�tulo \ref{chapter2} cont�m...

O cap�tulo \ref{chapter3} destina-se...

O cap�tulo \ref{chapter4} discute sobre...

Por fim, no cap�tulo \ref{chapter5} ser� feita uma conclus�o...

