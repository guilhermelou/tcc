
\chapter[Algoritmos]{Como Obter a Triangula\c{c}\~ao de Delaunay} \label{chapter4}

\section{Introdu\c{c}\~{a}o}

Introdu��o...

\section{Algoritmo de Inser\c{c}\~ao Randomizado}

Veremos abaixo o pseudo-c\'{o}digo desenvolvido do algoritmo a ser
implementado:

\begin{algorithm}
\caption{Algoritmo incremental}
\label{watson}
\begin{algorithmic}
\REQUIRE Um conjunto $P$ de $n$ pontos no plano
\ENSURE A Triangula\c{c}\~{a}o de Delaunay de P.
Seja ${\mathbf p}-1$, ${\mathbf p}-2$ e ${\mathbf p}-3$ um conjunto de tr\^{e}s pontos que formam uma grande tri\^{a}ngulos imagin\'{a}rio que contem $P$.
Inicializar a triangula\c{c}\~{a}o $\mathcal{T}$ que possui apenas um tri\^{a}ngulo simples
${\mathbf p}-1$, ${\mathbf p}-2$ e ${\mathbf p}-3$

\FOR {r $\Longleftarrow$ 1  \textbf{at\'{e}} n}
\STATE  Insere-se um novo ponto em $\mathcal{T}$
\STATE  Encontre o tri\^{a}ngulo ${p}_{i}{p}_{j}{p}_{k}$ que cont\'{e}m o ponto p.
\IF {p est\'{a} no interior do tri\^{a}ngulo ${p}_{i}{p}_{j}{p}_{k}$ }
\STATE Ligue o ponto p aos tr\^{e}s \textit{v\'{e}rtices} formados criando \textit{arestas} e dividindo 
${p}_{i}{p}_{j}{p}_{k}$ em tr\^{e}s tri\^{a}ngulos.
\STATE $\mathbf{flip}\left( p,\overline{p_{i}p_{j}}\right)$
\STATE $\mathbf{flip}\left( p,\overline{p_{j}p_{k}}\right)$
\STATE $\mathbf{flip}\left( p,\overline{p_{k}p_{i}}\right)$
\ELSE 
\STATE Adicione arestas de p at\'{e} $\mathbf{p}_{k}$ e ao terceiro \textit{v\'{e}rtice} $\mathbf{p}_{l}$ 
\STATE do outro tri\^{a}ngulo que \'{e} incidente a $\overline{p_{i}p_{j}}$, logo dividindo os dois 
\STATE tri\^{a}ngulos incidentes a $\overline{p_{i}p_{j}}$ em quatro.
\STATE $\mathbf{flip}\left( p,\overline{p_{i}p_{l}}\right)$
\STATE $\mathbf{flip}\left( p,\overline{p_{l}p_{j}}\right)$
\STATE $\mathbf{flip}\left( p,\overline{p_{j}p_{k}}\right)$
\STATE $\mathbf{flip}\left( p,\overline{p_{k}p_{i}}\right)$
\ENDIF
\ENDFOR
\STATE Remova os \textit{v\'{e}rtices} p-1, p-2 e p-3 e tamb\'{e}m todas \textit{arestas} 
\STATE do supertri\^{a}ngulo incidentes na triangula\c{c}\~{a}o 
\end{algorithmic}
\end{algorithm}


\begin{figure}[htbp]
  \begin{center}
    \leavevmode
    \caption{Ilustra\c{c}\~{a}o das atualiza\c{c}\~{o}es topol\'{o}gicas na 
	             inser\c{c}\~{a}o do primeiro ponto \textbf{V}.}
    \label{fig6_chapter6}
  \end{center}
\end{figure}



\subsection{Escolha adequada das coordenadas de um supertri\^angulo}

No algoritmo incremental de Lawson, citamos a necessidade de criar-se um grande tri\^angulo que contenha todos os pontos da triangula\c{c}\~{a}o, as coordenadas dos v\'{e}rtices do grande tri\^{a}ngulo segundo o criterio devem ser escolhidas como sendo os maiores valores poss\'{i}veis. Por\'em o ideal \'{e} que escolhamos n\'umeros adequados para que o supertri\^{a}ngulo contenha todos os pontos que ser\~{a}o processados e que n\~{a}o tenha coordenadas muito exageradas. Ent\~{a}o o que faremos ser\'{a} escolher os pontos ${\mathbf p}-1$, ${\mathbf p}
-2$ e ${\mathbf p}-3$ e modificar o teste feito para arestas ilegais, funcionar como se estiv\'essemos os escolhido com coordenadas bem grandes. Uma sugest\~{a}o feita em \cite{berg} \'{e} que a gera\c{c}\~{a}o de pontos seja feita no interior de um quadril\'{a}tero de lado $M$ e que o supertri\^{a}ngulo dever\'{a} envolver o quadril\'{a}tero tendo coordenadas de dist\^ancia $3M$ ( $[0, 3M]$, $[3M, 0]$, $[-3M, -3M]$ ), embora ao inv\'{e}s de $3M$ alguns autores sugerem o uso de uma dist\^ancia de $6M$ ( $[0, 6M]$, $[6M, 0]$, $[-6M, -6M]$ ) figura \ref{super_tri_3M}.
Uma outra sugest\~{a}o \'{e} utilizar um circulo de raio $M$, e gerar os pontos no interior do quadril\'{a}tero inscrito na circunfer\^{e}ncia segundo a figura \ref{super_tri_circ_3M}

\begin{figure}[htbp]
  \begin{center}
    \leavevmode
    \caption{Supertri\^angulo envolvendo o quadril\'atero com coordenadas $3M$.}
    \label{super_tri_3M}
  \end{center}
\end{figure}


\begin{figure}[htbp]
  \begin{center}
    \leavevmode
    \caption{Circunfer\^encia de raio  $3M$ inscrita no Supertri\^angulo envolve um quadril\'atero onde ser\~ao gerados os pontos.}
    \label{super_tri_circ_3M}
  \end{center}
\end{figure}

