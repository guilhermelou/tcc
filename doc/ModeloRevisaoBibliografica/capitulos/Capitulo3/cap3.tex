
\chapter{Triangula��o de Delaunay} \label{chapter3}

Em alguns casos, h\'{a} mais de uma \textit{Triangula\c{c}\~{a}o de Delaunay}
para um conjunto de pontos, estes casos s\~{a}o chamados de degenerados. Um
exemplo de caso degenerado \'{e} mostrado na Figura \ref{fig1_chapter5},
 onde h\'{a} quatro pontos localizados nos \textit{v\'{e}rtices}
de um quadrado. Neste caso h\'{a} um \'{u}nico \textit{v\'{e}rtice} de
Voronoi localizado no centro do quadrado, e est\'{a} associado com quatro
regi\~{o}es de Voronoi. Duas triangula\c{c}\~{o}es s\~{a}o poss\'{\i}veis
com este diagrama e ambas s\~{a}o v\'{a}lidas. Na pr\'{a}tica, os casos
degenerados devem ser resolvidos fazendo uma escolha entre as triangula\c{c}%
\~{o}es v\'{a}lidas.

\begin{figure}[htbp]
  \begin{center}
    \leavevmode
    \caption{Casos degenerados.} 
    \label{fig1_chapter5}
  \end{center}
\end{figure}

\begin{figure}[htbp]
  \begin{center}
    \leavevmode
    \caption{a) Conjunto de pontos distribuidos arbitrariamente no plano;
	     b) Fecho convexo do conjunto de pontos em  a)
	     c) Triangula\c{c}\~{a}o de Delaunay incidente sobre o conjunto de pontos a);
	     d) Diagrama de Voronoi do conjunto de pontos  a).}
    \label{fig2_chapter5}
  \end{center}
\end{figure}

Para ilustrar a base geom\'{e}trica da \textit{Triangula\c{c}\~{a}o de Delaunay} 
\'{e} conveniente considerar sua dupla geometria, o diagrama de Voronoi.
Considere um conjunto de pontos no plano como mostrado na figura \ref{fig2_chapter5} a). 
O diagrama de Voronoi associado com estes pontos \'{e} mostrado
pelas linhas tracejadas na figura \ref{fig2_chapter5} d). A \textit{Triangula\c{c}\~{a}o de Delaunay}
\'e mostrada na figura \ref{fig2_chapter5} c).

\begin{figure}[htbp]
  \begin{center}
    \leavevmode
    \caption{Combina\c{c}\~{a}o de todos os problemas geom\'etricos listados na figura \ref{fig2_chapter5}. O conjunto de pontos \'e exatamente o mesmo da figura \ref{fig2_chapter5} a).}
    \label{fig_combina}
  \end{center}
\end{figure}



\section{Defini\c{c}\~{o}es}

Seja $P=\{p_{1},p_{2},...,p_{n}\}$ um conjunto de $n$ $(n\geq3)$ pontos
distintos do plano pertencentes a $R^{n}$.

\begin{Def}
O Diagrama de Voronoi de $P$ \'{e} a subdivis\~{a}o do plano em $n$ 
c\'{e}lulas, para cada ponto em $P$, com a propriedade que um ponto $q$ pertence
a c\'{e}lula correspondente ao ponto $p_{i}$ se e somente se

\begin{equation}
\mathit{dist(\mathbf {q}, \mathbf{p}}_{i}\mathit{)<dist(\mathbf{q}, \mathbf{p}}_{j}\mathit{)}
\end{equation}

para cada $p_{j}\in P$  com $i\neq j$.

Denotamos o Diagrama de Voronoi por $V(P)$. A c\'{e}lula
que corresponde ao ponto $p_{i}$, denotamos por $V(p_{i})$.%
\end{Def}

\begin{Def} 
Consideremos dois pontos $p_{1}$%
e $p_{2}$. Definimos o bisector perpendicular do segmento 
$\overline{p_{1}p_{2}}$ como sendo $B(p_{1},p_{2})=B_{12}$. 
Ent\~{a}o cada ponto $x$ sobre $B_{12}$ est\'{a}
equidistante de $p_{1}$ e $p_{2}$.
\end{Def}



\section{Propriedades}

Veremos a seguir algumas caracter\'{\i}sticas importantes da \textit{%
Triangula\c{c}\~{a}o de Delaunay} que pode ser encontrada em \cite{musin}, \cite{orourke}. 
Sendo $P$ um conjunto de pontos e chamaremos de $D(P)$ a \textit{Triangula\c{c}\~{a}o de Delaunay} do
respectivo conjunto de pontos $P$:

\begin{itemize}
\item Se $P$ \'{e} um conjunto finito $\in R^{2}$ e seja abc 3 pontos que
comp\~{o}em um c\'{\i}rculo, abc \'{e} um tri\^{a}ngulo de Delaunay se o c%
\'{\i}rculo n\~{a}o conter nenhum outro ponto;

\item Entre todas as triangula\c{c}\~{o}es de um dado conjunto de pontos, a
Triangula\c{c}\~{a}o de Delaunay maximiza lexicograficamente o menor dos 
\^{a}ngulos de qualquer tri\^{a}ngulo;


\item Uma \textit{aresta} obedece \`{a} regra de Delaunay se:\end{itemize}


- Pertence a um \'{u}nico tri\^{a}ngulo e fica na fronteira da regi\~{a}o
convexa ou;

- Pertence a 2 tri\^{a}ngulos \textbf{abc }e \textbf{abd} sendo que \textbf{d 
}fica no exterior do c\'{\i}rculo determinado por \textbf{abc.}

A partir dessa postura citaremos alguns teoremas e
lemas importantes, inclusive alguns obtidos em Sibson\cite{green}, que far\~{a}o
uma pequena explana\c{c}\~{a}o entre as primitivas e a triangula\c{c}\~{a}o e
tamb\'{e}m destacar\'{a} caracter\'{\i}sticas relevante das propriedades da
triangula\c{c}\~{a}o.

\textbf{Teorema}\textit{(C\'{\i}rculo Circunscrito): A Circunfer\^{e}ncia
Circunscrita em qualquer tri\^{a}ngulo de Delaunay n\~{a}o cont\'{e}m nenhum
outro ponto da triangula\c{c}\~{a}o.}

\ \ \ \ \ \ \ \ \ \ \ 

\textbf{Corol\'{a}rio: }\textit{A pr\'{e}-Triangula\c{c}\~{a}o de Delaunay 
\'{e} n\~{a}o-degenerada, e deste modo \'{e} uma triangula\c{c}\~{a}o,
unicamente caracterizada pelos equi\^angulos locais. O resultado \'{e}
afirmado por Green e Sibson\cite{green} e a presente nota validou a sua afirma%
\c{c}\~{a}o. Green e Sibson descreveram um algoritmo para construir apenas
a Triangula\c{c}\~{a}o de Delaunay; este algoritmo foi implementado e
posteriormente executado com sucesso em grande escala - superando 10.000
dados apresentados - precisa de pouco espa\c{c}o para ser armazenado e \'{e}
muito r\'{a}pido. O presente teorema permitiu considerarmos este como um dos
meios de se construir equi\^angulos localizados nas triangula\c{c}\~{o}es para
interpola\c{c}\~{a}o. Diferente do algoritmo (apresentado por Lawson\cite{lawson})
que trabalhava diretamente em termos de \^{a}ngulos m\'{a}ximos-m\'{\i}%
nimos, crit\'{e}rios concorrentes para adquirir direto os recursos
aproximados no algoritmo Green-Sibson.}

\begin{description}
\item[1] $V$\textit{\ e }$D$\textit{\ s\~{a}o duais, isto \'{e}, para }$%
P,P^{^{\prime }}\subseteq S,$\textit{\ }$V(P)$\textit{\ \'{e} uma face de }$%
V(P^{^{\prime }})$\textit{\ se }$D(P)$\textit{\ \'{e} uma face de }$%
D(P^{^{\prime }})$.

\item[2] \textit{Se }$P\subseteq S,$\textit{\ }$V(P)$\textit{\ n\~{a}o
possuir\'{a} fronteira se cada lado de }$R$\textit{\ est\'{a} sobre a
fronteira do \textit{Fecho Convexo} de }$S.$
\end{description}

\begin{Lem}
: Seja $T_{L}$ e $T_{R}$ triangula\c{c}\~{o}es de Delaunay com um conjunto de $L$ e $R$ v\'{e}rtices. Ent\~{a}o podemos sempre construir uma triangula\c{c}\~{a}o de Delaunay $T$ para o conjunto $L$ $\cup $ $R$
tais que cada aresta de $T$ que n\~{a}o est\'{a} em $T_{L}$ ou $T_{R}$ tem um ponto final em $L$ e um em $R$.
\end{Lem}



\section{Como gerar uma Triangula\c{c}\~{a}o Delaunay}

Aqui ser� abordada...

\subsection{A t\'{e}cnica do \textit{flip-edge}}

Para verificar se o tri\^{a}ngulo satisfaz a regra de Delaunay primeiramente
passeamos pela lista duplamente encadeada, atrav\'{e}s das primitivas 
\textbf{pcw} e \textbf{pccw}, de \textit{arestas}(\textbf{e}), verificando se cada uma de suas \textit{faces}(f1,f2) possui algum \textit{v\'{e}rtice} no interior da circunfer\^{e}ncia que circunscreve a \textit{face }triangular. Observe a Figura \ref{fig3_chapter5} b) para melhor compreens\~{a}o.

\begin{figure}[htbp]
  \begin{center}
    \leavevmode
    \caption{a) Tri\^{a}ngulos cuja aresta comum entre eles n\~{a}o passou no teste do \textit{InC\'{i}rculo}; 
             b) C\'{i}rculo formado pelos pontos pertencentes aos dois tri\^{a}ngulos que revela um 
			 quarto ponto em seu interiores; 
			 c) Remo\c{c}\~{a}o da aresta ilegal; 
			 d) Ap\'{o}s o \textit{flip}(troca da aresta do quadril\'{a}tero) ambos tri\^{a}ngulos
			 satisfazem o crit\'{e}rio de Delaunay.}
    \label{fig3_chapter5}
  \end{center}
\end{figure}


\begin{figure}[htbp]
  \begin{center}
    \leavevmode
    \caption{a) Pelo fato de existirem tr\^{e}s pontos cocirculares $\mathbf{p}_{1}$, 
	$\mathbf{p}_{2}$ e $\mathbf{p}_{3}$, trata-se de um caso degenerado 
	do \textit{flip-edge} e deve ser tratado de maneira especial. b) como n\~ao se trata de uma regi\~ao convexa nesse tri\^angulo n\~ao se aplica a t\'ecnica do \textit{flip-edge}.}
    \label{fig4_chapter5}
  \end{center}
\end{figure}

\textbf{Descri\c{c}\~{a}o do pseudo-c\'{o}digo }\textit{flip-edge.}

\begin{algorithm}
\caption{flip-edge}
\label{flip}
\begin{algorithmic}
\REQUIRE Dois tri\^{a}ngulos que n\~{a}o passaram no teste do Inc\'{i}rculo.
\ENSURE A diagonal do quadril\'{a}tero formado pelos tri\^{a}ngulos trocada e o teste do Inc\'{i}rculo 
satisfeito plenamente.
\FOR {Os tri\^{a}ngulos que  n\~{a}o satisfazem o Inc\'{i}rculo}
\STATE  Remover uma \textit{aresta} uma \textit{face} e um \textit{loop}
\STATE  Realizar atualiza\c{c}\~{o}es topol\'{o}gicas 
\STATE  Criar uma nova \textit{aresta} na outra diagonal da \textit{face} do quadril\'{a}tero resultante
um novo \textit{loop} e uma nova \textit{face}.
\STATE  Atualizar novamente a topologia das \textit{arestas} da \textit{face} e do \textit{loop}.
\ENDFOR
\end{algorithmic}
\end{algorithm}

Com essa t\'{e}cnica conseguimos tri\^{a}ngulos que satisfazem o \textit{Crit%
\'{e}rio de Delaunay}, semelhante ao crit\'erio MaxMin(se\c{c}\~{a}o seguinte), com um n\'{u}mero reduzido de opera\c{c}\~{o}es. Assim poderemos tanto obter uma Triangula\c{c}\~{a}o de Delaunay a partir de uma triangula\c{c}\~{a}o arbitr\'{a}ria como tamb\'{e}m usando o algoritmo de inser\c{c}\~{a}o verificar durante a inser\c{c}\~{a}o dos pontos se os tri\^{a}ngulos s\~{a}o Delaunay.



\section{Maximizando os \^{a}ngulos m\'{\i}nimos.}

Seja $T$ uma triangula\c{c}\~{a}o de um conjunto de pontos $S$, e suponha
que ela tenha $m$ tri\^{a}ngulos. Consideremos os $3m$ \^{a}ngulos dos tri%
\^{a}ngulos em $T$, ordenados em ordem crescente. Seja $\alpha _{1},$ $%
\alpha _{2},\ldots ,$ $\alpha _{3m}$ o resultado da ordena\c{c}\~{a}o dos 
\^{a}ngulos, consequentemente, $\alpha _{i}\leq $ $\alpha _{j},$ para $i<j$
e chamaremos de $A(T)=(\alpha _{1},\alpha _{2},\ldots ,\alpha _{3m})$ o
vetor-\^{a}ngulo de $T$. Seja $T^{\prime }$ uma triangula\c{c}\~{a}o do mesmo
conjunto de pontos $S$, e seja $A\left( T^{\prime }\right) =\left( \alpha
_{1}^{\prime },\;\alpha _{2}^{\prime },\ldots ,\;\alpha _{3m}^{\prime
}\right) $ seu vetor-\^{a}ngulo. Dizemos que o vetor-\^{a}ngulo de $T$ \'{e}
maior que o de $T^{\prime }$ se $A(T)$ \'{e} lexicograficamente maior que $%
A(T^{\prime })$, ou, em outras palavras, se existe um \'{\i}ndice $i$ com $%
1\leq i\leq 3m$ de forma que:

\begin{equation}
\left\{
\begin{array}{l}
\alpha_{j} = \alpha_{j}^{\prime} 
\ \ \  j\leq i \\
\alpha_{i}> \alpha _{i}^{\prime }\\
\end{array}
\right.
\end{equation}

Denotamos como $A\left( T\right) >A\left( T^{\prime }\right) .$ Uma Triangula%
\c{c}\~{a}o de $T$ \'{e} chamada de \^{a}ngulo-\'{o}timo se $A\left(
T\right) \geq A\left( T^{\prime }\right) $ para todas as triangula\c{c}\~{o}%
es de $T^{\prime }$ de $S$. Note que uma triangula\c{c}\~{a}o que maximiza o
vetor-\^{a}ngulo para uma ordem lexicogr\'{a}fica tamb\'{e}m maximiza o
menor de seus \^{a}ngulos, e tal triangula\c{c}\~{a}o \'{e} chamada de 
\textit{globalmente equiangular}.

\textbf{Teorema: }\textit{Seja }$\mathbf{C}$ \textit{um c\'{\i}rculo, }$l$ \textit{%
uma linha cruzando }$\mathbf{C}$ \textit{nos pontos }$\mathbf{a}$ \textit{e} 
$\mathbf{b}$, \textit{e} $\mathbf{p}$, $\mathbf{q}$, $\mathbf{r}$ 
\textit{e} $\mathbf{s}$ \textit{pontos localizados no mesmo lado da linha }$%
l$\textit{. Suponha que }$\mathbf{p}$\textit{\ e }$\mathbf{q}$\textit{\ estejam sobre}
$\mathbf{C}$\textit{, que }$\mathbf{r}$\textit{\ esteja dentro de }$\mathbf{C}$, \textit{% 
e que} $\mathbf{s}$ \textit{esteja fora de} $\mathbf{C}$. Figura \ref{fig5_chapter5}. Ent\~{a}o:

\begin{equation}
\angle{\mathbf{arb}}>\angle{\mathbf{apb}}=\angle{\mathbf{aqb}}>\angle{\mathbf{asb}}.
\end{equation}

\begin{figure}[htbp]
  \begin{center}
    \leavevmode
    \caption{Maximiza\c{c}\~{a}o dos \^{a}ngulos m\'{i}nimos.}
    \label{fig5_chapter5}
  \end{center}
\end{figure}

Agora considerando uma \textit{aresta} $e=\overline{p_{i}p_{j}}$ de uma
triangula\c{c}\~{a}o $T$ de $S$. Se $e$ n\~{a}o \'{e} uma \textit{aresta}
pertencente ao \textit{Fecho Convexo}, ela \'{e} incidente a dois tri\^{a}ngulos $%
\triangle \left( p_{i}p_{j}p_{k}\right) $ e $\triangle \left(
p_{i}p_{j}p_{l}\right) .$ Se estes dois tri\^{a}ngulos formam um quadril\'{a}%
tero convexo, n\'{o}s podemos ter uma nova triangula\c{c}\~{a}o $T^{\prime }$
simplesmente fazendo o \textit{flip-edge }retirando $\overline{p_{i}p_{j}}$
e inserindo $\overline{p_{k}p_{l}}.$ A figura \ref{fig3_chapter5} ilustra a a%
\c{c}\~{a}o. Logo dizemos que uma \textit{aresta} \'{e} ilegal se conseguirmos fazer
o \textit{flip }nessa \textit{aresta}\textit{\ }aumentando o menor dos \^{a}%
ngulos.

\textbf{Lema: }\textit{Seja a aresta }$\overline{p_{i}p_{j}}$\textit{\
incidente aos tri\^{a}ngulos }$p_{i}p_{j}p_{k}$\textit{\ e }$p_{i}p_{j}p_{l}$%
\textit{, e seja }$C$\textit{\ o c\'{\i}rculo passando por }$p_{i},\;p_{j},$%
\textit{\ e}$\;p_{k}$\textit{. A aresta }$\overline{p_{i}p_{j}}$\textit{\ ser%
\'{a} ilegal se e somente se }$p_{l}$\textit{\ est\'{a} no inteiror de }$C$%
\textit{. Al\'{e}m disso, se os pontos }$p_{i},\;p_{j},\;p_{k},\;p_{l}$%
\textit{\ \ formam um quadril\'{a}tero convexo e n\~{a}o est\~{a}o em um c%
\'{\i}rculo comum, ent\~{a}o exatamente uma das arestas }$\overline{%
p_{i}p_{j}}$\textit{\ e }$\overline{p_{k}p_{l}}$\textit{\ \'{e} uma aresta
ilegal.} Figura \ref{fig3_chapter5}(b). 



\section{Aplica\c{c}\~{o}es Pr\'aticas da Triangula\c{c}\~{a}o de Delaunay}

Muitas aplica��es...

\begin{equation} \label{relations}
MST \subseteq RNG \subseteq GG \subseteq Delaunay
\end{equation}


\subsection{GG - Grabriel graph}

O grafo de Gabriel \'e um grafo ${\mathbf G}(V, E)$ que conecta um conjunto de pontos no plano euclidiano. Dois pontos $\mathbf{p}_1$ e $\mathbf{p}_2$ s\~ao conectados por uma aresta no grafo de Gabriel se o c\'irculo contendo o di\^ametro $\overline{{\mathbf p}_1{\mathbf p}_2}$ n\~ao possui nenhum outro ponto no seu interior figura \ref{fig_gabriel_circ} a), caso haja um terceiro ponto no interior da circunfer\^encia figura \ref{fig_gabriel_circ} b), a aresta n\~ao far\'a parte do grafo de Gabriel. Em qualquer outra dimens\~ao o grafo de Gabriel conecta dois pontos formando o di\^ametro de uma esfera vazia. O grafo de Gabriel foi apresentada por K. Ruben Gabriel e intitulada  por  Robert R. Sokal em 1969 \cite{sokal}. Esse grafo \'e um sub-grafo da Triangula\c{c}\~{a}o de Delaunay e com a triangula\c{c}\~{a}o dada pode ser obtido em tempo linear \cite{matula}. Esse gr\'afico cont\'em a MST e o RNG na inst\^ancia de um esqueleto-beta. Sua aplicabilidade tamb\'em se encontra na \'area de redes, na montagem de redes wireless. Um exemplo de grafo pode ser visto na figura \ref{fig_gabriel_graph}.


\begin{figure}[htbp]
  \begin{center}
    \leavevmode
    \caption{a) circunfer\^encia que cont\'em o di\^ametro de $\overline{{\mathbf p}_1{\mathbf p}_2}$, com um terceiro ponto ${\mathbf p}_3$ no exterior da circunfer\^encia. b) circunfer\^encia que cont\'em o di\^ametro de $\overline{{\mathbf p}_1{\mathbf p}_2}$, por\'em com um terceiro ponto ${\mathbf p}_3$ no interior da circunfer\^encia}
    \label{fig_gabriel_circ}
  \end{center}
\end{figure}


\begin{figure}[htbp]
  \begin{center}
    \leavevmode
    \caption{Gabriel Graph de um determinado conjunto de pontos em destaque sobre a Triangula\c{c}\~{a}o de Delaunay - linha pontilhadas}
    \label{fig_gabriel_graph}
  \end{center}
\end{figure}

