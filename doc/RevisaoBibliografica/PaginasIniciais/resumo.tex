\begin{resumo}

Elemento obrigat�rio, constitu�do de uma sequ�ncia de frases concisas 
e objetivas e n�o de uma simples enumera��o de t�picos. Deve conter de 
150 a 500 palavras, seguido logo abaixo das palavras-chave
(conforme NBR 6028).

Elemento obrigat�rio. Consiste em uma vers�o do resumo em idioma de 
divulga��o internacional (em ingl�s Abstract, em espanhol Resumen, 
em franc�s R�sum�). Tamb�m deve ser seguido de palavras-chave,

\begin{textblock}{12.1}(0,5)
{\bf palavras-chave:} palavras representativas do conte�do do trabalho, separadas entre si por ponto
(.) e finalizadas tamb�m por ponto. Devem aparecer logo abaixo do resumo, antecedidas da express�o 
``palavras-chave:''.
\end{textblock}

\end{resumo}

